% !TEX root = ./main.tex
% !TEX encoding = UTF-8 Unicode
% !TEX program = pdflatex
% !TeX spellcheck = it_IT

\chapter{Introduction}
\textit{"Yelp è un social network in cui le persone si scambiano pareri, opinioni e “dritte” sui posti migliori del luogo in cui abitano e di quelli in cui vanno per lavoro, viaggio o altri motivi."}\footnote{Claudia Resta - Community di Yelp}\\\\
La mission di questo elaborato è analizzare un dataset fornito da Yelp, contenente informazioni riguardanti differenti business, utenti ed un loro sottoinsieme di recensioni effettuate, al fine di individuare gli utenti più influenti della rete.\\
In particolare, quello della \textbf{Influence Maximization} è il problema di trovare un piccolo \textit{subset} di nodi in una social network tale da  massimizzare la diffusione di influenza (\textit{spread influence}).

\section{Dataset Originale}
Il dataset iniziale è fornito di 5 differenti files in formato "\textbf{json}":
\begin{itemize}
	\item \textbf{User}: contiene informazioni riguardanti gli utenti iscritti e le loro amicizie (circa 1.1M);
	\item \textbf{Business}: contiene informazioni riguardanti i business recensiti (circa 144K);
	\item \textbf{Review}: contiene le recensioni che gli utenti hanno effettuato per i business (circa 4.1M);
	\item \textbf{Checkin}: contiene informazioni riguardanti le visite effettuate nel tempo presso i differenti business;
	\item \textbf{Tip}: contiene informazioni riguardanti i suggerimenti che gli utenti hanno lasciato ai differenti business.
\end{itemize}

\'E stata effettuata un'analisi preliminare del problema andando direttamente ad utilizzare la piattaforma, al fine di comprenderne meglio le dinamiche di funzionamento.\\
In particolare, ogni utente iscritto possiede una propria pagina personale, sulla quale altri utenti possono lasciare differenti "\textit{complimenti}"