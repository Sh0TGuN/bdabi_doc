% !TEX root = ./main.tex
% !TEX encoding = UTF-8 Unicode
% !TEX program = pdflatex
% !TeX spellcheck = it_IT

\chapter{Studio preliminare del dataset}
Il dataset iniziale è fornito di 5 differenti files in formato "\textbf{json}":
\begin{itemize}
	\item \textbf{User}: contiene informazioni riguardanti gli utenti iscritti e le loro amicizie (circa 1.1M);
	\item \textbf{Business}: contiene informazioni riguardanti i business recensiti (circa 144K);
	\item \textbf{Review}: contiene le recensioni che gli utenti hanno effettuato per i business (circa 4.1M);
	\item \textbf{Checkin}: contiene informazioni riguardanti le visite effettuate nel tempo presso i differenti business;
	\item \textbf{Tip}: contiene informazioni riguardanti i suggerimenti che gli utenti hanno lasciato ai differenti business.
\end{itemize}

\'E stata effettuata un'analisi preliminare del problema andando direttamente ad
utilizzare la piattaforma, al fine di comprenderne meglio le dinamiche di
funzionamento.\\
In particolare, ogni utente iscritto possiede una propria
pagina personale, sulla quale altri utenti possono lasciare differenti
"\textit{complimenti}".\\
Ogni utente può lasciare una recensione ed un punteggio in
\textbf{stars} ad un business che ha visitato.\\
Ogni recensione, a sua volta, può essere giudicata dagli altri utenti con tre
differenti reazioni:
\begin{itemize}
	\item \textbf{Funny};
	\item \textbf{Useful};
	\item \textbf{Cool}.
\end{itemize}
Yelp utilizza anche un criterio per definire influenti o meno i suoi utenti iscritti,
assegnando il titolo di "\textit{elite}".\\

\section{Analisi in Power BI}
Al fine di calpire al meglio le caratteristiche dei dati e, soprattutto, come
utilizzarli allo scopo della \textbf{Influence Maximization}, si è deciso di
effettuare un'analisi preliminare utilizando il tool di BI
